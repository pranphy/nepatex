% Author : Prakash Gautam
% Date   : 11-02-2022 22:44:45
% vim: ai ts=4 sts=4 et sw=4 ft=tex
%
\documentclass[a4paper]{article}
\usepackage[margin=28mm]{geometry}
\usepackage{amsmath}
\usepackage{amssymb}
\usepackage{parskip}
\usepackage{graphicx}
\usepackage{nepali} % This is the package we are using.

\usepackage{fontspec}

\setmainfont[Renderer=HarfBuzz]{Laila}


\title{अब \LaTeX ~ नेपालीमा पनि}
\date{\today}
\author{ प्रकाश}



\begin{document}

    \maketitle

    \begin{abstract}
        यो सारांश हो। यसलाई अङ्ग्रेजीमा abstract भनिन्छ। यसले यो दस्तावेजमा भएका कुराहरूको सार सङ्क्षेपमा हामीलाई जानकारी दिन्छ। यसालई धेरै लामो पनि बनाउनु हुँदैन तर यसले पुरै दस्तावेजमा भएका कुराहरूको बारेमा छोटो र छिटो जानकारी दिन सक्नुपर्छ।
    \end{abstract}

    \tableofcontents
    \section{खण्ड}
        यो दस्तावेज nepali.sty प्रयोग गरेर बनाइएको हो। यसलाई हामी https://github.com/pranphy/nepatex.git बाट प्राप्त गर्न सक्छौँ।

        यहाँ तल एउटा समिकरण छ।

        \begin{align} \label{eq:trig-tan}
            \frac{\sin{\phi}}{\cos{\phi}} = \tan{\phi}
        \end{align}
        
        समिकरण (\ref{eq:trig-tan}) मा एउटा त्रिकोणमितिको समान्य सुत्र छ।

        मलाई मनपर्ने सुत्र यो हो।

        \begin{align*}
            G_{\mu\nu} + \Lambda g_{\mu\nu} = 8\pi T_{\mu\nu}
        \end{align*}

        यसलाई सापेक्षतावादको ब्यापक सिद्धान्त पनि भन्न सकिन्छ।

    \section{अन्य कुराहरू}

        \begin{enumerate}
            \item यो पहिलो कुरा हो।
            \item यो दोस्रो कुरा हो।
            \begin{enumerate}
                \item यहाँ भित्र  पनि छ। 
                \item अर्को भित्र  पनि छ। 
            \end{enumerate}
            \item अझै निरन्तर गर्न त सकिने नै भयो।
        \end{enumerate}

    \section{कान्तिपुरबाट सानो लेख}
        गोवामा एउटा मन्दिर थियो— मंगेश मन्दिर। मंग+ईश . मंगेश। मंगेश मन्दिरको वा देवताको नाउँमा त्यहाँ एउटा गाउँ बस्यो— मंगेशी । सम्भावना यो पनि छ कि, गाउँ मंगेशी पहिल्यैदेखि थियो । मंगेशी ग्राम--वासीहरूले कुनै कालमा त्यो मन्दिर बनाएको अथवा महादेवको मूर्ति प्रकट भएकाले, गाउँलेहरूले मन्दिर बनाए र पूजा गर्न थाले। कहिलेदेखि, यो ज्ञात छैन। मंगेशी गाउँका निवासी भन्थे, अति प्राचीनकालमा । त्यसका पुजारी हुन्थे, एक पुजारीको सेखपछि उसको सन्तान पुजारी बन्थ्यो । त्यसरी कुनै बखत एक पुजारीको एक पुत्र जन्म्यो— दीनानाथ ।

        \begin{table}[h!]
            \centering
            \caption{यहाँ तालिका छ तल हेर्नुस्}
            \begin{tabular}{lll}
                एक & दुई & तीन \\
                \hline \hline
                क & ख & ग\\
                \hline
            \end{tabular}
        \end{table}
    
        \begin{align}
            \int\limits_0^\infty x^{n-1} e^{-x} dx = \Gamma(n)
        \end{align}
    \section{चित्र भएको खण्ड}

        \begin{figure}[h!]
            \centering
            \includegraphics[width=.60\linewidth]{../images/logo.png}
            \caption{यो चित्र हाम्रो प्याकेजको आइकन हो।}
            \label{fig:pkg-icn}
        \end{figure}
        हामी हाम्रो यो प्याकेजको आइकन चित्र (\ref{fig:pkg-icn}) मा देख्न सक्छौँ।

\end{document}
